% !TEX encoding = UTF-8
\documentclass{article}

\usepackage{graphicx}
\usepackage{listings}
\usepackage{hyperref}
\usepackage{amsmath}
\usepackage{amssymb}
\usepackage[LGRx,T1]{fontenc}
\usepackage[utf8]{inputenc}
\usepackage[english,greek]{babel}

\newcommand{\en}[1]{\foreignlanguage{english}{#1}}

\begin{document}

\title{Προχωρημένα Θέματα Βάσεων Δεδομένων}
\author{Αλέξανδρος Δημήτριος Κέρος\\
	A.M.2008030109\\
	\en{LAB40624686}}
\date{}
\maketitle

\newpage
\tableofcontents

\newpage

%---------------------------------------------------------
\section{ΕΙΣΑΓΩΓΗ} \label{sec:Intro}

Σκοπός της παρούσας εργασίας είναι η υλοποίηση των τελεστών \emph{\en{Hash-Join}} και \emph{\en{Group-By (Count)}} επί του κατανεμημένου συστήματος αποθήκευσης και επεξεργασίας μεγάλου όγκου δεδομένων \emph{\en{Hadoop}}, τηρώντας το προγραμματιστικό μοντέλο \emph{\en{MapReduce}} για την επεξεργασία δεδομένων με χρήση παράλληλων, κατανεμημένων αλγορίθμων.

Η δομή της παρούσας αναφοράς είναι η ακόλουθη. Στο κεφάλαιο \ref{sec:Prelim} πραγματοποιείται συνοπτική αναφορά στο σύστημα \emph{\en{Hadoop}} και στο μοντέλο \emph{\en{MapReduce}}, όπως επίσης και στους αλγορίθμους \emph{\en{Hash Join}} και \emph{\en{GroupBy}}. Στο κεφάλαιο \ref{sec:Impl} περιγράφεται αναλυτικά η υλοποίηση των τελεστών (\emph{\en{HashJoin,GroupBy}}), καθώς και των επιπλέον λειτουργιών που ενσωματώθηκαν. Στο κεφάλαιο \ref{sec:ExpRes} αναλύεται η απόδοση των υλοποιηθέντων λειτουργιών όσον αφορά απαιτήσεις χρόνου και ορθότητας αποτελεσμάτων. Τέλος, στο κεφάλαιο \ref{sec:Eval} συνοψίζεται η παρούσα αναφορά και επισημαίνονται ορισμένες συμπερασματικές παρατηρήσεις.

%---------------------------------------------------------
\section{ΠΡΟΑΠΑΙΤΟΥΜΕΝΑ} \label{sec:Prelim}

\subsection{Μοντέλο και Σύστημα λογισμικού, \en{MapReduce \& Hadoop}} \label{subsec:HadoopMapReduce}
\subsubsection{\en{MapReduce}} \label{subsubsec:Mapreduce}

\emph{\en{MapReduce}} αποκαλείται το ``προγραμματικό μοντέλο, καθώς και η σχετική υλοποίηση για την επεξεργασία και την παραγωγή μεγάλου όγκου δεδομένων'' \cite{mapred_paper}, τα οποία αναπτύχθηκαν από τους μηχανικούς \en{Jeffrey Dean} και \en{Sanjay Ghemawat} της \en{Google}. Κίνητρο αποτέλεσε η αποδοτική επεξεργασία υπερμεγέθους όγκου δεδομένων σε εύλογο χρονικό διάστημα μέσω ενός παράλληλου και κατανεμημένου αφαιρετικού πλαισίου.

Το αφαιρετικό πλαίσιο βασίζεται στη δημιουργία απλών μονάδων επεξεργασίας, οι οποίες δηλώνονται ως \texttt{\en{map}} και \texttt{\en{reduce}} συναρτήσεις, απαλλάσσοντας τον προγραμματιστή από τα προβλήματα που ανακύπτουν από την κατανεμημένη εκτέλεση της εργασίας. Οι συναρτήσεις αυτές αναλαμβάνουν την επεξεργασία ζευγών \emph{κλειδί-τιμή} (\emph{\en{key-value pairs}}) για την παραγωγή ενός άλλου ζεύγους  \emph{κλειδί-τιμή} (\emph{\en{key-value}}).

Αναλυτικά, οι \texttt{\en{map}} συναρτήσεις αναλαμβάνουν τον μετασχηματισμό ενός αρχικού συνόλου ζευγών \emph{κλειδί-τιμή} σε ένα ενδιάμεσο ζεύγος \emph{κλειδί-τιμή}, τα οποία στη συνέχεια συνενώνονται ανά τιμή ενδιάμεσου κλειδιού και κατανέμονται στις \texttt{\en{reduce}} συναρτήσεις, οι οποίες τα επεξεργάζονται και  καταγράφουν το τελικό αποτέλεσμα (βλ. σχήμα \ref{fig:mapred-keyVals}, \ref{fig:mapred-graph} ). Προγράμματα γραμμένα κατά αυτόν τον τρόπο εκτελούνται παράλληλα από το πλαίσιο σε ένα σύνολο από υπολογιστές.

\begin{figure}[H]
\centering{\includegraphics[scale=0.6]{mrkeyval.png}
\caption{Μετατροπή ζευγών \emph{κλειδί-τιμή}(\emph{\en{key-value}}).\label{fig:mapred-keyVals}
}
}
\end{figure}

\begin{figure}[H]
\centering{\includegraphics[scale=0.35]{mrgraph.png}
\caption{Διαδικασία προγράμματος \en{MapReduce}.\label{fig:mapred-graph}
}
}
\end{figure}

\subsubsection{\en{Hadoop}} \label{subsubsec:Hadoop}

Το \emph{\en{Hadoop}} αποτελεί την ανοιχτού κώδικα υλοποίηση, βασισμένη σε \emph{\en{Java}}, του \emph{\en{MapReduce}} πλαισίου από το \emph{\en{Apache Software Foundation}}. Ουσιαστικά αποτελεί την συνένωση του \emph{\en{MapReduce}} πλαισίου, όπως παρουσιάζεται στη δημοσίευση \cite{mapred_paper}, με το Κατανεμημένο Σύστημα Αρχείων της \emph{\en{Google}} (\emph{\en{ Google File System [GFS]}}), όπως παρουσιάζεται στη δημοσίευση \cite{gfs_paper}, το οποίο υπό το πρόγραμμα του \emph{\en{Apache Foundation}} αποκαλείται \emph{\en{Hadoop Distributed File System [HDFS]}}. Με το σύστημα αυτό προσφέρονται λοιπόν τα κατάλληλα εργαλεία για την παράλληλη και κατανεμημένη επεξεργασία μεγάλου όγκου δεδομένων σε συστάδες υπολογιστών με αξιόπιστο και ασφαλή τρόπο, μέσω του μοντέλου \emph{\en{MapReduce}}. Η δομή του συστήματος παρουσιάζεται στο σχήμα \ref{fig:hadoop-arch}.

\begin{figure}[H]
\centering{\includegraphics[scale=0.5]{hadoopArch.png}
\caption{Δομή συστήματος \emph{\en{Hadoop}}.\label{fig:hadoop-arch}
}
}
\end{figure}

\subsection{Αλγόριθμοι και Τελεστές, \en{Hash Join \& Group By}}\label{subsec:HashJoinGroupBy}

\subsubsection{\en{Hash Join}}\label{subsubsec:HashJoin}

\subsubsection{\en{Group By}}\label{subsubsec:GroupBy}

%---------------------------------------------------------
\section{ΥΛΟΠΟΙΗΣΗ} \label{sec:Impl}

\subsection{\en{Hash Join}} \label{subsec:HJ}

\subsection{\en{Group By}} \label{subsec:GB}

\subsection{\en{Merge}} \label{subsec:MG}

\subsection{Επιπλέον Λειτουργίες} \label{subsec:Extras}

%---------------------------------------------------------
\section{Πειραματικά Αποτελέσματα} \label{sec:ExpRes}

%---------------------------------------------------------
\section{Συμπεράσματα} \label{sec:Eval}

μπλα μπλα

\addcontentsline{toc}{section}{Αναφορές}
	\begin{thebibliography}{9}
		\bibitem{n} \foreignlanguage{english}{\url{http://}}

	\end{thebibliography}
\end{document}