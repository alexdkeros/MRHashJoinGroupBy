% !TEX encoding = UTF-8
\documentclass{article}

\usepackage{graphicx}
\usepackage{listings}
\usepackage{hyperref}
\usepackage{amsmath}
\usepackage{amssymb}
\usepackage[LGRx,T1]{fontenc}
\usepackage[utf8]{inputenc}
\usepackage[english,greek]{babel}

\newcommand{\en}[1]{\foreignlanguage{english}{#1}}

\begin{document}

\title{Προχωρημένα Θέματα Βάσεων Δεδομένων}
\author{Αλέξανδρος Δημήτριος Κέρος\\
	A.M.2008030109\\
	\foreignlanguage{english} {LAB40624686}}
\date{}
\maketitle

\newpage
\tableofcontents

\newpage

%---------------------------------------------------------
\section{ΕΙΣΑΓΩΓΗ} \label{sec:Intro}

Σκοπός της παρούσας εργασίας είναι η υλοποίηση των τελεστών \emph{\en{Hash-Join}} και \emph{\en{Group-By (Count)}} επί του κατανεμημένου συστήματος αποθήκευσης και επεξεργασίας μεγάλου όγκου δεδομένων \emph{\en{Hadoop}}, τηρώντας το προγραμματιστικό μοντέλο \emph{\en{MapReduce}} για την επεξεργασία δεδομένων με χρήση παράλληλων, κατανεμημένων αλγορίθμων.

Η δομή της παρούσας αναφοράς είναι η ακόλουθη. Στο κεφάλαιο \ref{sec:Prelim} πραγματοποιείται συνοπτική αναφορά στο σύστημα \emph{\en{Hadoop}} και στο μοντέλο \emph{\en{MapReduce}}. Στο κεφάλαιο \ref{sec:Impl} περιγράφεται αναλυτικά η υλοποίηση των τελεστών (\emph{\en{Hadoop,GroupBy}}), καθώς και των επιπλέον λειτουργιών που ενσωματώθηκαν. Στο κεφάλαιο \ref{sec:ExpRes} αναλύεται η απόδοση των υλοποιηθέντων λειτουργειών όσον αφορά απαιτήσεις χρόνου και ορθότητας αποτελεσμάτων. Τέλος, στο κεφάλαιο \ref{sec:Eval} συνοψίζεται η παρούσα αναφορά και επισημαίνονται ορισμένες συμπερασματικές παρατηρήσεις.

%---------------------------------------------------------
\section{ΠΡΟΑΠΑΙΤΟΥΜΕΝΑ} \label{sec:Prelim}

\subsection{\en{Hadoop}} \label{subsec:Hadoop}

\subsection{\en{MapReduce}} \label{subsec:Mapreduce}

%---------------------------------------------------------
\section{ΥΛΟΠΟΙΗΣΗ} \label{sec:Impl}

\subsection{\en{Hash Join}} \label{subsec:HJ}

\subsection{\en{Group By}} \label{subsec:GB}

\subsection{\en{Merge}} \label{subsec:MG}

\subsection{Επιπλέον Λειτουργίες} \label{subsec:Extras}

%---------------------------------------------------------
\section{Πειραματικά Αποτελέσματα} \label{sec:ExpRes}

%---------------------------------------------------------
\section{Συμπεράσματα} \label{sec:Eval}

μπλα μπλα

\addcontentsline{toc}{section}{Αναφορές}
	\begin{thebibliography}{9}
		\bibitem{n} \foreignlanguage{english}{\url{http://}}

	\end{thebibliography}
\end{document}