%!TEX encoding = UTF-8
\documentclass{article}

\usepackage{graphicx}
\usepackage{listings}
\usepackage{hyperref}
\usepackage{amsmath}
\usepackage{amssymb}
\usepackage[LGRx,T1]{fontenc}
\usepackage[utf8]{inputenc}
\usepackage[english,greek]{babel}
\usepackage{algorithm}
\usepackage{algpseudocode}

\newcommand{\en}[1]{\foreignlanguage{english}{#1}}

\begin{document}

\title{Προχωρημένα Θέματα Βάσεων Δεδομένων}
\author{Αλέξανδρος Δημήτριος Κέρος\\
	A.M.2008030109\\
	\en{LAB40624686}}
\date{}
\maketitle

\newpage
\tableofcontents

\newpage

%---------------------------------------------------------
\section{ΕΙΣΑΓΩΓΗ} \label{sec:Intro}

Σκοπός της παρούσας εργασίας είναι η υλοποίηση των τελεστών \emph{\en{Hash-Join}} και \emph{\en{Group-By (Count)}} επί του κατανεμημένου συστήματος αποθήκευσης και επεξεργασίας μεγάλου όγκου δεδομένων \emph{\en{Hadoop}}, τηρώντας το προγραμματιστικό μοντέλο \emph{\en{MapReduce}} για την επεξεργασία δεδομένων με χρήση παράλληλων, κατανεμημένων αλγορίθμων.

Η δομή της παρούσας αναφοράς είναι η ακόλουθη. Στο κεφάλαιο \ref{sec:Prelim} πραγματοποιείται συνοπτική αναφορά στο σύστημα \emph{\en{Hadoop}} και στο μοντέλο \emph{\en{MapReduce}}, όπως επίσης και στους αλγορίθμους \emph{\en{Hash Join}} και \emph{\en{GroupBy}}. Στο κεφάλαιο \ref{sec:Impl} περιγράφεται αναλυτικά η υλοποίηση των τελεστών (\emph{\en{HashJoin,GroupBy}}), καθώς και των επιπλέον λειτουργιών που ενσωματώθηκαν. Στο κεφάλαιο \ref{sec:ExpRes} αναλύεται η απόδοση των υλοποιηθέντων λειτουργιών όσον αφορά απαιτήσεις χρόνου και ορθότητας αποτελεσμάτων. Τέλος, στο κεφάλαιο \ref{sec:Eval} συνοψίζεται η παρούσα αναφορά και επισημαίνονται ορισμένες συμπερασματικές παρατηρήσεις.

%---------------------------------------------------------
\section{ΠΡΟΑΠΑΙΤΟΥΜΕΝΑ} \label{sec:Prelim}

\subsection{Μοντέλο και Σύστημα λογισμικού, \en{MapReduce \& Hadoop}} \label{subsec:HadoopMapReduce}
\subsubsection{\en{MapReduce}} \label{subsubsec:Mapreduce}

\emph{\en{MapReduce}} αποκαλείται το ``προγραμματικό μοντέλο, καθώς και η σχετική υλοποίηση για την επεξεργασία και την παραγωγή μεγάλου όγκου δεδομένων'' \cite{mapred_paper}, τα οποία αναπτύχθηκαν από τους μηχανικούς \en{Jeffrey Dean} και \en{Sanjay Ghemawat} της \en{Google}. Κίνητρο αποτέλεσε η αποδοτική επεξεργασία υπερμεγέθους όγκου δεδομένων σε εύλογο χρονικό διάστημα μέσω ενός παράλληλου και κατανεμημένου αφαιρετικού πλαισίου.

Το αφαιρετικό πλαίσιο βασίζεται στη δημιουργία απλών μονάδων επεξεργασίας, οι οποίες δηλώνονται ως \texttt{\en{map}} και \texttt{\en{reduce}} συναρτήσεις, απαλλάσσοντας τον προγραμματιστή από τα προβλήματα που ανακύπτουν από την κατανεμημένη εκτέλεση της εργασίας. Οι συναρτήσεις αυτές αναλαμβάνουν την επεξεργασία ζευγών \emph{κλειδί-τιμή} (\emph{\en{key-value pairs}}) για την παραγωγή ενός άλλου ζεύγους  \emph{κλειδί-τιμή} (\emph{\en{key-value}}).

Αναλυτικά, οι \texttt{\en{map}} συναρτήσεις αναλαμβάνουν τον μετασχηματισμό ενός αρχικού συνόλου ζευγών \emph{κλειδί-τιμή} σε ένα ενδιάμεσο ζεύγος \emph{κλειδί-τιμή}, τα οποία στη συνέχεια συνενώνονται ανά τιμή ενδιάμεσου κλειδιού και κατανέμονται στις \texttt{\en{reduce}} συναρτήσεις, οι οποίες τα επεξεργάζονται και  καταγράφουν το τελικό αποτέλεσμα (βλ. σχήμα \ref{fig:mapred-keyVals}, \ref{fig:mapred-graph} ). Προγράμματα γραμμένα κατά αυτόν τον τρόπο εκτελούνται παράλληλα από το πλαίσιο σε ένα σύνολο από υπολογιστές.

\begin{figure}[H]
\centering{\includegraphics[scale=0.6]{mrkeyval.png}
\caption{Μετατροπή ζευγών \emph{κλειδί-τιμή}(\emph{\en{key-value}}).\label{fig:mapred-keyVals}
}
}
\end{figure}

\begin{figure}[H]
\centering{\includegraphics[scale=0.35]{mrgraph.png}
\caption{Διαδικασία προγράμματος \en{MapReduce}.\label{fig:mapred-graph}
}
}
\end{figure}

\subsubsection{\en{Hadoop}} \label{subsubsec:Hadoop}

Το \emph{\en{Hadoop}} αποτελεί την ανοιχτού κώδικα υλοποίηση, βασισμένη σε \emph{\en{Java}}, του \emph{\en{MapReduce}} πλαισίου από το \emph{\en{Apache Software Foundation}}. Ουσιαστικά αποτελεί την συνένωση του \emph{\en{MapReduce}} πλαισίου, όπως παρουσιάζεται στη δημοσίευση \cite{mapred_paper}, με το Κατανεμημένο Σύστημα Αρχείων της \emph{\en{Google}} (\emph{\en{ Google File System [GFS]}}), όπως παρουσιάζεται στη δημοσίευση \cite{gfs_paper}, το οποίο υπό το πρόγραμμα του \emph{\en{Apache Foundation}} αποκαλείται \emph{\en{Hadoop Distributed File System [HDFS]}}. Με το σύστημα αυτό προσφέρονται λοιπόν τα κατάλληλα εργαλεία για την παράλληλη και κατανεμημένη επεξεργασία μεγάλου όγκου δεδομένων σε συστάδες υπολογιστών με αξιόπιστο και ασφαλή τρόπο, μέσω του μοντέλου \emph{\en{MapReduce}}. Η δομή του συστήματος παρουσιάζεται στο σχήμα \ref{fig:hadoop-arch}.

\begin{figure}[H]
\centering{\includegraphics[scale=0.5]{hadoopArch.png}
\caption{Δομή συστήματος \emph{\en{Hadoop}}.\label{fig:hadoop-arch}
}
}
\end{figure}

\subsection{Αλγόριθμοι και Τελεστές, \en{Hash Join \& Group By}}\label{subsec:HashJoinGroupBy}

\subsubsection{\en{Hash Join \& Parallel Hash Join}}\label{subsubsec:HashJoin}

Ο \emph{\en{Hash Join}} τελεστής αποτελεί έναν αλγόριθμο συνένωσης σχέσεων επί κάποιας ισότητας ή ανισότητας στοιχείου στήλης μέσω πίνακα κατακερματισμού. Ο βασικός αλγόριθμος παρουσιάζεται στους αλγόριθμους \ref{algo:SHJ} και \ref{algo:GHJ}.
\selectlanguage{english}

\begin{algorithm}
\caption{Simple Hash Join}
\label{algo:SHJ}
\begin{algorithmic}[1]
\Procedure{build phase}{}
\For{each tuple $i \in R$}
\State add $i$ to in-memory hash table $H$
\EndFor
\EndProcedure
\Procedure{probe phase}{}
\For{each tuple $j \in S$}
\State probe $H$ for matching tuples $I$
\For{each matching tuple $i \in I$}
\State output$(i,j)$
\EndFor
\EndFor
\EndProcedure
\end{algorithmic}
\end{algorithm}

\begin{algorithm}
\caption{GRACE Hash Join}
\label{algo:GHJ}
\begin{algorithmic}[1]
\Procedure{build phase}{}
\For{each tuple $i \in R$}
\State add $i$ to hash bucket $R_i$ using $h$ hash function
\EndFor
\For{each tuple $j \in S$}
\State add $j$ to hash bucket $S_i$ using $h$ hash function
\EndFor
\EndProcedure

\Procedure{probe phase}{}
\For{each bucket $i$}
\For{each tuple $r \in R_i$}
\State insert into hash table $H$ using $h'$
\EndFor
\For{each tuple $s \in S_i$}
\State probe $H$ using $h'$
\State output joined tuples
\EndFor
\EndFor
\EndProcedure
\end{algorithmic}
\end{algorithm}

\selectlanguage{greek}

Ο παράλληλος \emph{\en{Hash Join}} αλγόριθμος αποσκοπεί στον διαχωρισμό των σχέσεων για ταυτόχρονη εκτέλεση του τελεστή σε πλήθος μηχανημάτων. Κάθε μηχάνημα εκτελεί ένα μέρος της συνένωσης και στο τέλος τα επιμέρους αποτελέσματα συμπτύσσονται. Σκοπός είναι ο περιορισμός του χρόνου εκτέλεσης μέσω κατανομής εργασίας σε πληθώρα επεξεργαστών.
 
\subsubsection{\en{Group By \& Parallel Group By}}\label{subsubsec:GroupBy}

Ο \emph{\en{Group By}} τελεστής αποτελεί έναν αλγόριθμο ομαδοποίησης και υπολογισμού κάποιας πράξης συνάθροισης επί ενός επιθυμητού συνόλους στηλών μιας σχέσης, όπως παρουσιάζεται στον αλγόριθμο \ref{algo:GB}.

\selectlanguage{english}
\begin{algorithm}
\caption{Group By}
\label{algo:GB}
\begin{algorithmic}[1]
\Procedure{Group By}{}
\For{each tuple $i \in R$}
\State add $i$ to $h(groupByColumns)$ bucket of hash table $H$
\EndFor
\For{each bucket $b \in H$}
\State compute aggregate over $b$ and output
\EndFor
\EndProcedure
\end{algorithmic}
\end{algorithm}
\selectlanguage{greek}

Ο παράλληλος \emph{\en{Group By}} αλγόριθμος εκμεταλεύεται τον κατακερματισμό της σχέσης (σύμφωνα με τον αλγόριθμο \ref{algo:GB}) για τον παράλληλο υπολογισμό των συναθροιστικών συναρτήσεων.

%---------------------------------------------------------
\section{ΥΛΟΠΟΙΗΣΗ} \label{sec:Impl}

Η εκτέλεση του \emph{\en{MapReduce Hash Join Group By}} τελεστή πραγματοποιείται μέσω της εντολής \\ \texttt{\en{hadoop jar your\_jar.jar MRHashJoinGroup.class path\_to\_left\_table path\_to\_right\_table
name\_of\_join\_column groupBy\_columns\_comma\_separated num\_of\_machines}}\\
 και αποτελείται από δύο \en{Hadoop jobs} στην περίπτωση που χρησιμοποιηθεί \en{file system merger} ή τρία \en{Hadoop jobs} στην περίπτωση που χρησιμοποιηθεί \en{MapReduce Merger} (βλ. \ref{subsec:MG}).
 
Παραμετροποίηση πρόσθετων λειτουργιών (\ref{subsec:MG},\ref{subsec:Extras}) δίνεται μέσω του αρχείου \texttt{\en{mapred-HJG.xml}}.

\subsection{\en{Hash Join}} \label{subsec:HJ}

Για την υλοποίηση του \emph{\en{Hash Join}} τελεστή χρησιμοποιήθηκε μια παράλληλη εκδοχή του \emph{\en{GRACE Hash Join}} αλγορίθμου (\ref{algo:GHJ}).

Αρχικά, οι \texttt{\en{map}} συναρτήσεις εξάγουν ως κλειδί το στοιχείο επί του οποίου πραγματοποιείται η συνένωση σε συνδυασμό με ένα αναγνωριστικό της σχέσης από την οποία προήλθε η κάθε γραμμή. Για τον λόγο αυτό χρησιμοποιήθηκε η κλάση \texttt{\en{TextPair.java}} \cite{textpair}. Ως τιμή εξάγεται ολόκληρη η γραμμή.

Στη συνέχεια πραγματοποιείται:
\begin{itemize}
\item \en{partitioning} (\texttt{\en{Partitioner}}) με βάση το πρώτο στοιχείο του κλειδιού (το στοιχείο της συνένωσης), ώστε να εξασφαλίζεται η συγκέντρωση των γραμμών με κοινά στοιχεία προς συνένωση στον ίδιο \en{Reducer}. 
\item \en{grouping}  (\texttt{\en{GroupComparator}}) με βάση το δεύτερο στοιχείο του κλειδιού (το αναγνωριστικό της σχέσης), ώστε να εξασφαλίζεται η  προτεραιότητα μεταξύ των σχέσεων για τις \en{build} και \en{probe} φάσεις του αλγορίθμου.
\item \en{sorting}  (\texttt{\en{KeyComparator}}) με βάση και τα δύο στοιχεία του κλειδιού για την ταξινόμησή τους.

Τέλος, οι \texttt{\en{reduce}} συναρτήσεις αναλαμβάνουν την εκτέλεση του \emph{\en{GRACE Hash Join}} αλγορίθμου στο \emph{τμήμα} των σχέσεων που έχουν λάβει ως είσοδο.

\end{itemize}
\begin{figure}[H]
\hspace*{-4.2cm}
\centering{\includegraphics[scale=0.45]{MRhashjoin.jpeg}
\caption{\en{MapReduce Hash Join} υλοποίηση.\label{fig:MRHJ}
}
}
\end{figure}


\subsection{\en{Group By}} \label{subsec:GB}

Για την υλοποίηση του \emph{\en{Group By}} τελεστή χρησιμοποιήθηκε μια παράλληλη εκδοχή του αλγορίθμου \ref{algo:GB}.
Αρχικά, οι \texttt{\en{map}} συναρτήσεις εξάγουν ως κλειδί την συνένωση των στοιχείων τα οποία επιθυμούμε να συναθροίσουμε. Ως τιμή εξάγεται ολόκληρη η γραμμή της σχέσης.

Κατα το \en{Shuffle and Sort} στάδιο χρησιμοποιούνται οι προκαθορισμένες κλάσεις.

Τέλος, οι \texttt{\en{reduce}} συναρτήσεις αναλαμβάνουν την συνάθροιση (εδώ \texttt{\en{count}}) των επιθυμητών τιμών αξιοποιώντας τη δομή μιας \emph{\en{MapReduce}} εργασίας, κατά την οποία ως είσοδο σε μια τέτοια συνάρτηση δίνεται μία δομή που περιέχει το σύνολο των τιμών που αντιστοιχούν σε ένα συγκεκριμένο κλειδί. 

\begin{figure}[H]
\hspace*{-4.2cm}
\centering{\includegraphics[scale=0.45]{MRgroupby.jpeg}
\caption{\en{MapReduce Group By} υλοποίηση.\label{fig:MRGB}
}
}
\end{figure}

\subsection{\en{Merge}} \label{subsec:MG}


\begin{figure}[H]
\hspace*{-4.2cm}
\centering{\includegraphics[scale=0.45]{MRMerger.jpeg}
\caption{\en{MapReduce Merger} υλοποίηση.\label{fig:MRMerger}
}
}
\end{figure}
\subsection{Επιπλέον Λειτουργίες} \label{subsec:Extras}

%---------------------------------------------------------
\section{Πειραματικά Αποτελέσματα} \label{sec:ExpRes}

%---------------------------------------------------------
\section{Συμπεράσματα} \label{sec:Eval}

μπλα μπλα

\addcontentsline{toc}{section}{Αναφορές}
	\begin{thebibliography}{9}
		\bibitem{n} \foreignlanguage{english}{\url{http://}}

	\end{thebibliography}
\end{document}